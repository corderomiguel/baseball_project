%%%%%%%%%%%%%%%%%%%%%%%%%%%%%%%%%%%%%
%%%%%%%%%% Plantilla del Team para el proyecto final 
%%%%%%%%%%%%%%%%%%%%%%%%%%%%%%%%%%%%%
%%%%%%%%%% Nueva clase personalizada
\documentclass{staprojteamusb}
%%%%%%%%%% Nuevo estilo personalizado
\usepackage{staprojteamusbsty}

\addbibresource{referencias.bib}
%%%%%%%%%%%%%%%%%%%%%%%%%%%%%%%%%%%
%%%%%%%%%% Variables de Markdown
%%%%%%%%%%%%%%%%%%%%%%%%%%%%%%%%%%%
%%%%%%%%%% Encabezado 
\titulo{Análisis estadístico sobre una base de datos de beísbol.}

\autor{
		Luis Riera\\
	Universidad Simón Bolívar \\
	Caracas, Venezuela \\
	\texttt{\href{mailto:16-10976@usb.ve}{\nolinkurl{16-10976@usb.ve}}} \\
	 \And
		Eduardo Gavazut\\
	Universidad Simón Bolívar \\
	Caracas, Venezuela \\
	\texttt{\href{mailto:13-10524@usb.ve}{\nolinkurl{13-10524@usb.ve}}} \\
	 \And
		Miguel Cordero\\
	Universidad Simón Bolívar \\
	Caracas, Venezuela \\
	\texttt{\href{mailto:15-10326@usb.ve}{\nolinkurl{15-10326@usb.ve}}} \\
	}
\fecha{8 de abril de 2022}
\resumen{En este documento se realiza un prueba de hipótesis sobre la
variable X2 o tasa de bateo para determinar si el promedio es menor que
0.3 .}

\palabrasc{Prueba de hipótesis, tasa de bateo, variable X2}
%%%%%%%%%%%% Resto de las variables propias de R eso viene por defecto de R

%%%%%%%% Pandoc Estilo  de los codigos
\usepackage{color}
\usepackage{fancyvrb}
\newcommand{\VerbBar}{|}
\newcommand{\VERB}{\Verb[commandchars=\\\{\}]}
\DefineVerbatimEnvironment{Highlighting}{Verbatim}{commandchars=\\\{\}}
% Add ',fontsize=\small' for more characters per line
\usepackage{framed}
\definecolor{shadecolor}{RGB}{248,248,248}
\newenvironment{Shaded}{\begin{snugshade}}{\end{snugshade}}
\newcommand{\AlertTok}[1]{\textcolor[rgb]{0.94,0.16,0.16}{#1}}
\newcommand{\AnnotationTok}[1]{\textcolor[rgb]{0.56,0.35,0.01}{\textbf{\textit{#1}}}}
\newcommand{\AttributeTok}[1]{\textcolor[rgb]{0.77,0.63,0.00}{#1}}
\newcommand{\BaseNTok}[1]{\textcolor[rgb]{0.00,0.00,0.81}{#1}}
\newcommand{\BuiltInTok}[1]{#1}
\newcommand{\CharTok}[1]{\textcolor[rgb]{0.31,0.60,0.02}{#1}}
\newcommand{\CommentTok}[1]{\textcolor[rgb]{0.56,0.35,0.01}{\textit{#1}}}
\newcommand{\CommentVarTok}[1]{\textcolor[rgb]{0.56,0.35,0.01}{\textbf{\textit{#1}}}}
\newcommand{\ConstantTok}[1]{\textcolor[rgb]{0.00,0.00,0.00}{#1}}
\newcommand{\ControlFlowTok}[1]{\textcolor[rgb]{0.13,0.29,0.53}{\textbf{#1}}}
\newcommand{\DataTypeTok}[1]{\textcolor[rgb]{0.13,0.29,0.53}{#1}}
\newcommand{\DecValTok}[1]{\textcolor[rgb]{0.00,0.00,0.81}{#1}}
\newcommand{\DocumentationTok}[1]{\textcolor[rgb]{0.56,0.35,0.01}{\textbf{\textit{#1}}}}
\newcommand{\ErrorTok}[1]{\textcolor[rgb]{0.64,0.00,0.00}{\textbf{#1}}}
\newcommand{\ExtensionTok}[1]{#1}
\newcommand{\FloatTok}[1]{\textcolor[rgb]{0.00,0.00,0.81}{#1}}
\newcommand{\FunctionTok}[1]{\textcolor[rgb]{0.00,0.00,0.00}{#1}}
\newcommand{\ImportTok}[1]{#1}
\newcommand{\InformationTok}[1]{\textcolor[rgb]{0.56,0.35,0.01}{\textbf{\textit{#1}}}}
\newcommand{\KeywordTok}[1]{\textcolor[rgb]{0.13,0.29,0.53}{\textbf{#1}}}
\newcommand{\NormalTok}[1]{#1}
\newcommand{\OperatorTok}[1]{\textcolor[rgb]{0.81,0.36,0.00}{\textbf{#1}}}
\newcommand{\OtherTok}[1]{\textcolor[rgb]{0.56,0.35,0.01}{#1}}
\newcommand{\PreprocessorTok}[1]{\textcolor[rgb]{0.56,0.35,0.01}{\textit{#1}}}
\newcommand{\RegionMarkerTok}[1]{#1}
\newcommand{\SpecialCharTok}[1]{\textcolor[rgb]{0.00,0.00,0.00}{#1}}
\newcommand{\SpecialStringTok}[1]{\textcolor[rgb]{0.31,0.60,0.02}{#1}}
\newcommand{\StringTok}[1]{\textcolor[rgb]{0.31,0.60,0.02}{#1}}
\newcommand{\VariableTok}[1]{\textcolor[rgb]{0.00,0.00,0.00}{#1}}
\newcommand{\VerbatimStringTok}[1]{\textcolor[rgb]{0.31,0.60,0.02}{#1}}
\newcommand{\WarningTok}[1]{\textcolor[rgb]{0.56,0.35,0.01}{\textbf{\textit{#1}}}}

%%%%%%%%% Estilo de las listas sin salto de linea
\providecommand{\tightlist}{%
	\setlength{\itemsep}{0pt}\setlength{\parskip}{0pt}}


%%%%%%%%%%%% Si hay paquetes por incluir en header-include
%%%%%%%%%%%%%%%%%%%%%%%%%%%%%%%%%%%%%%%%%%%%%%%%%%%%%%%%%
\begin{document}
	
	
	\maketitle
	
	%%%%%%%%%%%% Si hay cosas que incluir en include-before
		%%%%%%%%%%%%%%%%%%%%%%%%%%%%%%%%%%%%%%%%%%%%%%%%%%%%%%%%%
	
	%%%%%%%%%%%%% Inicio del documento
	
	\hypertarget{pruebas-sobre-la-tasa-de-bateo}{%
 \section{Pruebas sobre la tasa de
 bateo}\label{pruebas-sobre-la-tasa-de-bateo}}

 Se desea probar con un nivel de significancia de \(\alpha=0.05\), que
 el promedio de bateo es inferior a \(0.300\).

 Como hipótesis nula \(H_{0}\), supongamos que la media de bateo,
 \(\overline{X1}\), es igual a \(0.3\). Y como hipótesis alternativa,
 \(H_{a}\), que el promedio de bateo es superior a \(0.3\),
 \(\overline{X1}>0.3\).

 Suponiendo que los datos presentan una distribución normal, podemos
 aplicar el comando \texttt{t.test}.

 Con este función, se obtuvo que el valor para el estadístico \(t\) es
 \(-23.811\), con \(44\) grados libertad. Como el \(p-valor\) es
 bastante alto, de hecho es igual \(0,9976\) (que representa un
 \(99.76\%\)), se cumple que \(\alpha=0.05<99.76\) y por lo tanto la
 hipótesis alternativa se rechaza, mas aún, se rechaza para todo nivel
 de significancia porque se necesita un valor para \(\alpha\) más alto
 que el \(p-valor\) para rechazar la hipótesis nula.

 Se afirma entonces, con total seguridad, que la tasa de bateo es
 inferior a \(0.300\).

 \begin{center}\rule{0.5\linewidth}{0.5pt}\end{center}

\begin{Shaded}
\begin{Highlighting}[]
\CommentTok{\# Inicializamos la librería que permite leer archivos xlsx}
\FunctionTok{library}\NormalTok{(readxl)}
\CommentTok{\# Asignamos a una variable la información almacenada en el archivo}
\NormalTok{Baseball }\OtherTok{\textless{}{-}} \FunctionTok{read\_excel}\NormalTok{(}\StringTok{"\textasciitilde{}/GitHub/data/Baseball.xlsx"}\NormalTok{)}
\CommentTok{\# Mostramos las primeras 5 entradas}
\FunctionTok{head}\NormalTok{(Baseball, }\AttributeTok{n=}\DecValTok{5}\NormalTok{)}
\end{Highlighting}
\end{Shaded}

\begin{verbatim}
## # A tibble: 5 x 6
##      X1    X2    X3    X4    X5    X6
##   <dbl> <dbl> <dbl> <dbl> <dbl> <dbl>
## 1 0.283 0.144 0.049 0.012 0.013 0.086
## 2 0.276 0.125 0.039 0.013 0.002 0.062
## 3 0.281 0.141 0.045 0.021 0.013 0.074
## 4 0.328 0.189 0.043 0.001 0.03  0.032
## 5 0.29  0.161 0.044 0.011 0.07  0.076
\end{verbatim}

\begin{Shaded}
\begin{Highlighting}[]
\NormalTok{X1}\OtherTok{\textless{}{-}}\NormalTok{ Baseball}\SpecialCharTok{$}\NormalTok{X1}
\end{Highlighting}
\end{Shaded}

 \hypertarget{pruebas-sobre-la-tasa-de-bateo-1}{%
 \subsection{Pruebas sobre la tasa de
 bateo}\label{pruebas-sobre-la-tasa-de-bateo-1}}

 Se desea probar con un nivel de significancia de \(\alpha=0.05\), que
 el promedio de bateo es inferior a \(0.300\).

 Como hipótesis nula \(H_{0}\), supongamos que la media de bateo,
 \(\overline{X1}\), es igual a \(0.3\). Y como hipótesis alternativa,
 \(H_{a}\), que el promedio de bateo es superior a \(0.3\),
 \(\overline{X1}>0.3\).

 Suponiendo que los datos presentan una distribución normal, podemos
 aplicar el comando \texttt{t.test}.

 \hypertarget{section}{%
 \subsection{}\label{section}}

\begin{Shaded}
\begin{Highlighting}[]
\FunctionTok{t.test}\NormalTok{(X1, }\AttributeTok{alternative =} \StringTok{"greater"}\NormalTok{, }\AttributeTok{mu=}\FloatTok{0.3}\NormalTok{, }\AttributeTok{conf.level =} \FloatTok{0.95}\NormalTok{)}
\end{Highlighting}
\end{Shaded}

\begin{verbatim}
## 
##  One Sample t-test
## 
## data:  X1
## t = -2.9779, df = 44, p-value = 0.9976
## alternative hypothesis: true mean is greater than 0.3
## 95 percent confidence interval:
##  0.2694453       Inf
## sample estimates:
## mean of x 
## 0.2804667
\end{verbatim}

 Con este función, se obtuvo que el valor para el estadístico \(t\) es
 \(-23.811\), con \(44\) grados libertad. Como el \(p-valor\) es
 bastante alto, de hecho es igual \(0,9976\) (que representa un
 \(99.76\%\)), se cumple que \(\alpha=0.05<99.76\) y por lo tanto la
 hipótesis alternativa se rechaza, mas aún, se rechaza para todo nivel
 de significancia porque se necesita un valor para \(\alpha\) más alto
 que el \(p-valor\) para rechazar la hipótesis nula.

 Se afirma entonces, con total seguridad, que la tasa de bateo es
 inferior a \(0.300\).
	
	%%%%%%%%%%%%% Fin del documento
	
	%%%%%%%%%%%% Inicio de la bibliografia
	
	\printbibliography
	
	
	%%%%%%%%%%%%% Fin de la bibliografia
	
	%%%%%%%%%%%%%%%%% Si hay cosas que incluir en include-after
		%%%%%%%%%%%%%%%%%%%%%%%%%%%%%%%%%%%%%%%%%%%%%%%%%%%%%%%%%%%%
	
\end{document}
