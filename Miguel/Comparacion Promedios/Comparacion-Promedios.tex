%%%%%%%%%%%%%%%%%%%%%%%%%%%%%%%%%%%%%
%%%%%%%%%% Plantilla del Team para el proyecto final 
%%%%%%%%%%%%%%%%%%%%%%%%%%%%%%%%%%%%%
%%%%%%%%%% Nueva clase personalizada
\documentclass{staprojteamusb}
%%%%%%%%%% Nuevo estilo personalizado
\usepackage{staprojteamusbsty}

\addbibresource{referencias.bib}
%%%%%%%%%%%%%%%%%%%%%%%%%%%%%%%%%%%
%%%%%%%%%% Variables de Markdown
%%%%%%%%%%%%%%%%%%%%%%%%%%%%%%%%%%%
%%%%%%%%%% Encabezado 
\titulo{Análisis estadístico sobre una base de datos de beísbol.}

\autor{
		Eduardo Gavazut\\
	Universidad Simón Bolívar \\
	Caracas, Venezuela \\
	\texttt{\href{mailto:13-10524@usb.ve}{\nolinkurl{13-10524@usb.ve}}} \\
	 \And
		Luis Riera\\
	Universidad Simón Bolívar \\
	Caracas, Venezuela \\
	\texttt{\href{mailto:16-10976@usb.ve}{\nolinkurl{16-10976@usb.ve}}} \\
	 \And
		Miguel Cordero\\
	Universidad Simón Bolívar \\
	Caracas, Venezuela \\
	\texttt{\href{mailto:15-10326@usb.ve}{\nolinkurl{15-10326@usb.ve}}} \\
	}
\fecha{8 de abril de 2022}
\resumen{En este documento se realiza un prueba de hipótesis sobre la
variable X2 o tasa de bateo para determinar si el promedio es menor que
0.3 .}

\palabrasc{Prueba de hipótesis, tasa de bateo, variable X2}
%%%%%%%%%%%% Resto de las variables propias de R eso viene por defecto de R


%%%%%%%%% Estilo de las listas sin salto de linea
\providecommand{\tightlist}{%
	\setlength{\itemsep}{0pt}\setlength{\parskip}{0pt}}


%%%%%%%%%%%% Si hay paquetes por incluir en header-include
%%%%%%%%%%%%%%%%%%%%%%%%%%%%%%%%%%%%%%%%%%%%%%%%%%%%%%%%%
\begin{document}
	
	
	\maketitle
	
	%%%%%%%%%%%% Si hay cosas que incluir en include-before
		%%%%%%%%%%%%%%%%%%%%%%%%%%%%%%%%%%%%%%%%%%%%%%%%%%%%%%%%%
	
	%%%%%%%%%%%%% Inicio del documento
	
	\hypertarget{pruebas-sobre-la-tasa-de-bateo}{%
 \section{Pruebas sobre la tasa de
 bateo}\label{pruebas-sobre-la-tasa-de-bateo}}

 Se desea probar con un nivel de significancia de \(\alpha=0.05\), que
 el promedio de bateo es inferior a \(0.300\).

 Como hipótesis nula \(H_{0}\), supongamos que la media de bateo,
 \(\overline{X1}\), es igual a \(0.3\). Y como hipótesis alternativa,
 \(H_{a}\), que el promedio de bateo es superior a \(0.3\),
 \(\overline{X1}>0.3\).

 Suponiendo que los datos presentan una distribución normal, podemos
 aplicar el comando \texttt{t.test}.

 Con este función, se obtuvo que el valor para el estadístico \(t\) es
 \(-23.811\), con \(44\) grados libertad. Como el \(p-valor\) es
 bastante alto, de hecho es igual \(0,9976\) (que representa un
 \(99.76\%\)), se cumple que \(\alpha=0.05<99.76\) y por lo tanto la
 hipótesis alternativa se rechaza, mas aún, se rechaza para todo nivel
 de significancia porque se necesita un valor para \(\alpha\) más alto
 que el \(p-valor\) para rechazar la hipótesis nula.

 Se afirma entonces, con total seguridad, que la tasa de bateo es
 inferior a \(0.300\).
	
	%%%%%%%%%%%%% Fin del documento
	
	%%%%%%%%%%%% Inicio de la bibliografia
	
	\printbibliography
	
	
	%%%%%%%%%%%%% Fin de la bibliografia
	
	%%%%%%%%%%%%%%%%% Si hay cosas que incluir en include-after
		%%%%%%%%%%%%%%%%%%%%%%%%%%%%%%%%%%%%%%%%%%%%%%%%%%%%%%%%%%%%
	
\end{document}
